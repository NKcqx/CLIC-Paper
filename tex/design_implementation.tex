\section{SYSTEM DESIGN AND IMPLEMENTATION}
In this section, we describe the design and implementation of CLIC, including the support for agile platform integration, environment management, and the supports for developing cross-platform work- flows. 
The prototype of CLIC is built on Kubernetes, with the mod- ules and tasks running in Docker containers.

\subsection{Operator Mapping and Management}
CLIC provides a logical operator set that is a union of operators from multiple domains. 
For instance, the batch processing operator subset consists of MAP, FLATMAP, SORT, FILTER, etc., 
and the linear algebra subset consists of DOT PRODUCT, MULTIPLY, TRANSPOSE, and so on. 
A logical operator describes the operator name, input parameters, return value, algebra type, and related optimizing information. 
The information of all operators is stored in a configuration server and provided to developers by PlanBuilder. 
A physical operator contains information including the name of the physical function and data types on the platform. 
Each logical operator needs to be mapped to a physical operator for execution, i.e., assigning a platform to an operator.

\begin{center}
    \begin{tabular}{ c c c }
        Logical Operator & Platform & Physical Operator \\
        \multirow{3}{2em}{PCA} & Tensorflow & pca() \\
        %  & PyTorch & low_rank() \\
        %  & Spark ML & PCA() \\
        \multirow{2}{4em}{Map} & Spark & map() \\
         & JavaStream  & map() \\
        \multirow{3}{4em}{Word2Vec} & PyTorch & Word2Vec() \\
         & Spark ML & Word2Vev() \\
         & Gensim & Word2Vec() \\
    \end{tabular}
\end{center}

According to the workflow and workload, different physical operators can be chosen for deployment. 
We maintain an operator mapping table that contains one-to-many mappings between logical operators and physical operators. 
Table 1 exemplifies the mappings of three logical operators, i.e., PCA, Map, and Word2Vec. 
As listed in the table, Map has the implementation on Spark and JavaStream, while PCA and Word2Vec can be mapped to machine learning platforms including PyTorch and Tensorflow. 
When a logical operator does not have an equivalent native implementation on a platform, developers can implement it and update the mapping table, such as the customized Word2Vec operator in PyTorch. 
In the configuration server, the operator mapping table is maintained with the support for insert, delete, and replace operations.

\subsection{Image Management}
CLIC facilitates the integration of new platforms by maintaining them as images. 
The metadata of each image is stored in the image table on the configuration server, including Dockerfile, platform version, entry-point, and all available physical operators. 
In the image table, images of the same platform are stored together, and each image contains the platform of a different version. 
With the physical plan of a workflow being split into tasks on different platforms, a proper image is located and used for each task. 
When there exists different versions of the same image, the latest version that contains the required physical operators is selected.

To integrate a new platform, the image needs to be built with the corresponding information being updated in the image table.
If a new operator of a platform is developed, besides rebuilding the image and updating its information in the image table, 
the information of the new operator also needs to be updated in the operator mapping table.


\subsection{Supports for Building Cross-Platform Workflows}
Building a cross-platform data science workflow demands a set of functionalities and supports. 
In this section, we introduce the data model conversion, UDF passing and dynamic control flows in CLIC.


\textbf{Data Model Conversion} 
There can be significant differences between data models of different platforms.
For example, an array’s dimensions implicitly order cells while is different from a Table (SQL tuples) [34]; 
The graph model can perform better on relationship queries than the Table due to the inherent structural differences [27, 31]. 
These differences make it hard to construct a unified data model that can be used by all operators. 
Therefore, we bridge the gaps between data models by implementing a series of model conversion algorithms. 
The toMatrix operator in line 5 of Listing 1 is one of the conversion algorithms that converts a concrete word to a dense vector using the skip-gram [29] algorithm. 
Note that we provide essential conversion algorithms in our model conversion library, such as the skip-gram, and expose interfaces to develop conversions concerning business logic.

\textbf{UDF}  Due to the decoupled architecture, a user-defined function (UDF) is passed to a platform driver through RPCs. 
In CLIC, a UDF instance is firstly serialized and embedded in the logical plan by the Planbuilder. 
It is deserialized by the Executor inside the underlying platform to drive the execution. 
One tricky problem is when the UDF is implemented in a different lanugage with the underlying platform. 
CLIC currently supports translating UDFs between Python and Java using a 3rd-party tool.

\textbf{Dynamic Control Flow} We introduce control flow operators including SWITCH, NEXT ITERATION, and EXIT to help build workflows with iterations. 
The SWITCH primitive takes an if-statement as the input parameter, initial iteration flag values as input data, and returns a boolean value that directs the executor. 
The NEXT ITERATION operator takes iteration flags as input data and forwards them to the head of the next iteration block, e.g. SWITCH operator. 
EXIT operator indicates the end of an iteration and connects to the next operator in the workflow.
A simple workflow with iteration is shown in Figure 5 where $i$ is the iteration flag, $i < 10$ is the if-statement, and Map operator is the iteration body.

\subsection{Optimizing Logical Plan}
Optimizing the logical plan of a workflow can bring notable performance improvement [17, 22].
In CLIC, we implement optimizers as microservices, and two currently supported optimizers are illustrated as follows.

\textbf{Align Layout}
Access patterns (row-, column-, or block-wise) of linear algebra operators can significantly impact ML workflow performance [39]. 
For example, the matrix multiplication operator access the left matrix in a row-wise pattern and access the right matrix in a column-wise pattern. 
Changing the sparse matrix layout to match the access pattern can significantly improve the performance [25]. 
CLIC identifies the pattern of the sparse matrix multiplication and implicitly inserts a layout conversion operator.

\textbf{Broadcast}
Some binary operations like JOIN and MATRIX MULTIPLY need to frequently send data batches between nodes when input data is distributed stored. 
However, when one of the data is small enough to fit in memory, such as multiplying a matrix with a vector or joining a big table with a tiny one, the coordinating overheads can be eschewed by broadcasting the small one to all nodes so that they can be computed locally [8, 26].
CLIC identifies this pattern and implicitly inserts the BROADCAST operator.

